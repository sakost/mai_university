\section{Тест производительности}


% \begin{longtable}{|p{7.5cm}|p{7.5cm}|}
%     \hline
%     \rowcolor{lightgray}
%     \multicolumn{2}{|c|} {Длина ключа от 1 до 3}\\
%     \hline
%     168616&PATRICIA\\
%     \hline
%     239682&Встроенная структура map\\
%     \hline
% \end{longtable}

Наивное решение:
\begin{lstlisting}[language=c++]
    #include <algorithm>
    #include <iostream>
    #include <string>
    #include <vector>

    using namespace std;

    int main() {
        int n, m;
        cin >> n >> m;
        int count = 0;
        for (long long i(n); i > 0ll; --i) {
            auto s1 = to_string(i);
            auto s2 = to_string(n);
            if (i % m == 0 &&
            lexicographical_compare(s1.begin(), s1.end(), s2.begin(), s2.end())) {
                ++count;
            }
        }
        cout << count << endl;
    }
\end{lstlisting}

Сам замер:

\begin{alltt}
    sakost@sakost-pc ~/university/2 course/diskran/lab7
    $ time ./cmake-build-debug/naive << EOL
    10704020 59
    EOL
    13260
    ./cmake-build-debug/naive <<<'10704020 59'  1,86s user 0,00s system 99% cpu 1,865 total
    sakost@sakost-pc ~/university/2 course/diskran/lab7
    $ time ./cmake-build-debug/lab7 << EOL
    10704020 59
    EOL
    13260
    ./cmake-build-debug/lab7 <<<'10704020 59'  0,01s user 0,00s system 9% cpu 0,075 total

    sakost@sakost-pc ~/university/2 course/diskran/lab7
    $ time ./cmake-build-debug/naive << EOL
    107040200 59
    EOL
    132585
    ./cmake-build-debug/naive <<<'107040200 59'  15,34s user 0,00s system 99% cpu 15,346 total
    sakost@sakost-pc ~/university/2 course/diskran/lab7
    $ time ./cmake-build-debug/lab7 << EOL
    107040200 59
    EOL
    132585
    ./cmake-build-debug/lab7 <<<'107040200 59'  0,01s user 0,00s system 78% cpu 0,009 total
    sakost@sakost-pc ~/university/2 course/diskran/lab7
    $


\end{alltt}

По такому небольшому бенчмарку сразу видно отставание наивного алгоритма, причем в разы.


\pagebreak

