\section{Описание}
Требуется решить задачу методом динамического программирования. \\
\texttt{Динамическое программирование — это когда у нас есть задача, которую непонятно как решать, и мы разбиваем ее на меньшие задачи, которые тоже непонятно как решать. (с) А.Кумок} \cite{neerc_ifmo}
\\
Динамическое программирование -- метод программирования, при котором мы разбиваем исходную задачу на подзадачи и решаем эти "маленькие" подзадачи в каком-либо цикле.
\\
Данный метод применим к задачам с оптимальной подструктурой, выглядящим как набор перекрывающихся
подзадач, сложность которых чуть меньше исходной: в этом случае время вычислений можно значительно сократить.
Как правило, чтобы решить поставленную задачу, требуется решить отдельные части задачи(подзадачи), после чего объединить
решения подзадач в одно общее решение. Часто многие из этих подзадач одинаковые или похожи друг на друга.
Подход динамического программирования состоит в том, чтобы решить каждую отдельную задачу только один раз,
сократив, тем самым, количество вычислений. Это особенно полезно в случаях, когда число повторяющихся подзадач эксоненциально велико.
Этапы построения алгоритма решения задач динамическим программированием:
\begin{itemize}
	\item Описать структуру оптимального решения.
	\item Составить рекурсивное решение для нахождения оптимального решения.
	\item Вычислить значения, соответствующего оптимальному решению, методом восходящего анализа.
	\item Непосредственное нахождение оптимального решения из полученной на предыдущих этапах информации.
\end{itemize}

\pagebreak

\section{Исходный код}
Идея решения данной задачи основывается на довольно простом наблюдении:
На каждый разряд(в десятичной системе счисления) приходится ровно $ n/m - (10^{|str(n)|-1} - 1) / m $ чисел, которые меньше $n$ лексикографически и делятся на $m$, где $n$ -- само число, $m$ -- число из условия,
$n/m$ -- сколько в принципе чисел делятся на $m$(которые меньше $n$), а $(10 ^ {\lfloor{}log_{10}(n)\rfloor{} - 1} - 1) / m$ -- кол-во чисел, делящихся на $m$, но больше $n$ лексикографически(для одного старшего разряда). Причем в данном выражении учитываются и числа меньших разрядов(то есть они тоже вычитаются).
\\
То есть итоговое решение это сумма предыдущего ответа(для предыдущего разряда) и приведенной выше формулы для данного разряда. Необходимо повторять данный алгоритм, пока $n > 0$.
Файл \texttt{main.cpp}:
\begin{lstlisting}[language=C++]
#include <iostream>
#include <vector>
#include <string>
#include <cmath>
#include <algorithm>


int main() {
    std::int64_t n, m;
    std::cin >> n >> m;

    std::vector<std::int64_t> dp;
    dp.resize(std::to_string(n).size()+1);
    dp[0] = -bool(n%m==0);

    std::int64_t i(1);
    while(n > 0){
        dp[i] = n/m - ((std::int64_t)std::pow(10, std::to_string(n).size()-1) - 1) / m + dp[i-1];
        n /= 10;
        ++i;
    }

    std::cout << std::max(dp.back(), (std::int64_t)0) << std::endl;
}
\end{lstlisting}


\pagebreak

\section{Консоль}
\begin{alltt}
	sakost@sakost-pc ~/university/2 course/diskran/lab7
	$ cmake -S . -B cmake-build-debug
	-- The C compiler identification is GNU 11.1.0
	-- The CXX compiler identification is GNU 11.1.0
	-- Detecting C compiler ABI info
	-- Detecting C compiler ABI info - done
	-- Check for working C compiler: /sbin/cc - skipped
	-- Detecting C compile features
	-- Detecting C compile features - done
	-- Detecting CXX compiler ABI info
	-- Detecting CXX compiler ABI info - done
	-- Check for working CXX compiler: /sbin/c++ - skipped
	-- Detecting CXX compile features
	-- Detecting CXX compile features - done
	-- Configuring done
	-- Generating done
	-- Build files have been written to: /home/sakost/university/2 course/diskran/lab7/cmake-build-debug
	sakost@sakost-pc ~/university/2 course/diskran/lab7
	$ cmake --build cmake-build-debug --target lab7
	[ 50%] Building CXX object CMakeFiles/lab7.dir/main.cpp.o
	[100%] Linking CXX executable lab7
	[100%] Built target lab7
	sakost@sakost-pc ~/university/2 course/diskran/lab7
	$ ./cmake-build-debug/lab7
	42 3
	11
	sakost@sakost-pc ~/university/2 course/diskran/lab7
	$


	
\end{alltt}
\pagebreak

