\section{Выводы}

Динамическое программирование -- частый метод в решении в, на первый взгляд, не решаемых (быстро) задач.
Данный метод позволяет построить радикально ускоренную версию алгоритма.
Динамическое программирование может быть применимо практически везде, где есть какие-либо перекрывающиеся подзадачи, которые можно выделить из главной задачи.
В частности, как мне кажется, данный подход имеет место быть в сложных параллельных вычислениях. Например вычислении выхода разных нейронных сетей или других алгоритмов машинного обучения(поскольку там много перекрывающихся подзадач).
В данной лабораторной работе я также укрепил свои знания в подходе динамического программирования.
Стоит упомянуть, что ДП -- это скорее подход к построению алгоритмов, а не просто конкретный алгоритм.
\pagebreak
