\section{Описание}
Требуется написать реализацию алгоритма карманной сортировки. 

Как сказано в \cite{neerc_wiki}: \enquote{карманная сортировка -- алгоритм сортировки, основанный на предположении о равномерном распределении входных данных.}.

Идея карманной сортировки состоит в том, чтобы разбить исходные возможные значения в массиве на $n$ равных интервалов,
где $n$ -- количество элементов в массиве, а затем пройтись по массиву, перемещая в созданные интервалы(называемые карманами), 
элементы, которые в них попадают, а карманы, в свою очередь, отсортировать обычной сортировкой вставки.

\pagebreak

\section{Исходный код}
Написание кода состояло из нескольких этапов:
\begin{enumerate}
	\item Прописать объявления основных функций, классов, методов(функций-членов) и полей класса
	\item Написать прототип функции $main$, а именно:
	\begin{enumerate}
		\item Ввод-вывод данных
		\item Запуск сортировки
	\end{enumerate}
	\item Реализовать основные функции и функции-члены для работы с векторами
	\item Реализовать сортировку вставками
	\item Реализовать карманную сортировку(используя сортировку вставками)
	\item Написать Makefile
\end{enumerate}

На каждой непустой строке входного файла располагается пара \enquote{ключ-значение}, поэтому создадим новые 
структуры $TKey$ и $TValue$, которые будут являться псевдонимами для входных типов данных(это сделано для универсальности кода), также, т.к. тип $NPair::TPair<NSort::TKey, NSort::TValue>$
является слишком длинным, я решил сделать для него псевдоним $TData$.

\paragraph{}
Файл \texttt{main.cpp}:
\begin{lstlisting}[language=C++]
	#include <iostream>
	#include <algorithm>
	
	#include "vector.h"
	#include "pair.h"
	#include "sort.h"
	
	
	using TData = NPair::TPair<NSort::TKey, NSort::TValue>;
	
	int main() {
		std::ios::sync_with_stdio(false);
		std::cin.tie(nullptr);
		std::cout.tie(nullptr);
	
		NVector::TVector<TData> vector;
	
		NSort::TKey key;
		NSort::TValue value;
		while(std::cin >> key >> value){
			vector.PushBack(NPair::TPair<NSort::TKey, NSort::TValue>(key, value));
		}
	
		NSort::BucketSort(vector);
	
		for (int i = 0; i < vector.Size(); ++i) {
			std::cout << vector[i].First << ' ' << vector[i].Second << std::endl;
		}
	
		return 0;
	}
\end{lstlisting}

Файл \texttt{vector.h}:
\begin{lstlisting}[language=C++]
	#pragma once

	#include <stdexcept>
	#include <cstring>
	
	
	namespace NVector {
		const int DEFAULT_CAPACITY_MULTIPLIER = 2;
	
		template<typename T>
		class TVector {
		public:
			TVector();
	
			explicit TVector(size_t newSize, const T &defaultValue = T());
	
			[[nodiscard]] size_t Size() const;
	
			[[nodiscard]] bool Empty() const;
	
			[[maybe_unused]] T *Begin() const;
	
			[[maybe_unused]] T *End() const;
	
	
			TVector(const TVector &other);
	
			TVector &operator=(const TVector &other);
	
			~TVector();
	
			void Clear();
	
			[[maybe_unused]] void PushBack(T &element);
	
			void PushBack(const T &&element);
	
			const T &At(size_t index) const;
	
			T &At(size_t index);
	
			const T &operator[](size_t index) const;
	
			T &operator[](size_t index);
	
		private:
			size_t Len{};
			size_t Capacity{};
			T *Arr;
		};
	}
\end{lstlisting}
Файл \texttt{pair.h}:
\begin{lstlisting}[language=C++]
//
// Created by sakost on 02.10.2020.
//

#ifndef LAB1_PAIR_H
#define LAB1_PAIR_H


namespace NPair {

    template<typename F, typename S>
    class TPair {
    public:
        TPair();

        TPair(F first, S second);

        TPair(const TPair &other);

        ~TPair()= default;

        bool operator<(const TPair &other) const;

        TPair &operator=(const TPair &other);


        F First;
        S Second;
    };
}


#endif //LAB1_PAIR_H
\end{lstlisting}

Файл \texttt{sort.h}
\begin{lstlisting}[language=C++]
	//
// Created by sakost on 02.10.2020.
//

#ifndef LAB1_SORT_H
#define LAB1_SORT_H

#include <cinttypes>
#include <limits>
#include <cassert>

#include "vector.h"
#include "pair.h"

namespace NSort{
    using TKey = std::uint64_t;
    using TValue = std::uint64_t;

    NPair::TPair<TKey, TValue> MaxElement(const NVector::TVector<NPair::TPair<TKey, TValue>> &vector);
    NPair::TPair<TKey, TValue> MinElement(const NVector::TVector<NPair::TPair<TKey, TValue>> &vector);

    void InsertionSort(NVector::TVector<NPair::TPair<TKey, TValue>> &vector);

    void BucketSort(NVector::TVector<NPair::TPair<TKey, TValue>> &vector);
}


#endif //LAB1_SORT_H
\end{lstlisting}

Файл \texttt{benchmark.cpp}
\begin{lstlisting}[language=C++]
//
// Created by sakost on 03.10.2020.
//


#include "pair.h"
#include "vector.h"
#include "sort.h"

#include <iostream>
#include <cstdint>
#include <chrono>
#include <algorithm>
#include <random>

int main() {
    std::ios_base::sync_with_stdio(false);
    std::cin.tie(nullptr);
    std::cout.tie(nullptr);

    NVector::TVector<NPair::TPair<NSort::TKey , NSort::TValue>> v;
    NPair::TPair<NSort::TKey , NSort::TValue> pair;


    auto start = std::chrono::steady_clock::now();
    NSort::TKey key;
    NSort::TValue value;
    while (std::cin >> key >> value) {
        pair.First = key;
        pair.Second = value;
        v.PushBack(pair);
    }
    auto finish = std::chrono::steady_clock::now();
    auto dur = finish - start;
    std::cerr << "input " << std::chrono::duration_cast<std::chrono::milliseconds>(dur).count() << " ms" << std::endl;

    start = std::chrono::steady_clock::now();
    NSort::BucketSort(v);
    finish = std::chrono::steady_clock::now();
    dur = finish - start;
    std::cerr << "custom bucket sort " << std::chrono::duration_cast<std::chrono::milliseconds>(dur).count() << " ms" << std::endl;

    std::random_device rd;
    std::mt19937 g(rd());
    std::shuffle(v.Begin(), v.End(), g);

    start = std::chrono::steady_clock::now();
    std::stable_sort(v.Begin(), v.End());
    finish = std::chrono::steady_clock::now();

    dur = finish - start;
    std::cerr << "stable sort from std " << std::chrono::duration_cast<std::chrono::milliseconds>(dur).count() << " ms" << std::endl;

    start = std::chrono::steady_clock::now();
    for (size_t i = 0; i < v.Size(); i++) {
        std::cout << v[i].First << ' ' << v[i].Second << '\n';
    }
    finish = std::chrono::steady_clock::now();
    dur = finish - start;
    std::cerr << "output " << std::chrono::duration_cast<std::chrono::milliseconds>(dur).count() << " ms" << std::endl;

    return 0;
}

\end{lstlisting}


Таблица 1, описывающая структуру программы:
\begin{longtable}{|p{7.5cm}|p{7.5cm}|}
\hline
\rowcolor{lightgray}
\multicolumn{2}{|c|} {vector.h}\\
\hline
class TVector&Класс, реализующий тип данных \enquote{вектор}\\
\hline
TVector()&Конструктор по умолчанию\\
\hline
explicit TVector(size\_t newSize, const T \&defaultValue = T())&Конструктор с заданием размера или и размера, и элемента по умолчанию\\
\hline
size\_t Size() const&Функция-член, возвращающая размер вектора\\
\hline
bool Empty() const&Функция-член, возвращающая значение, которое указывает, пустой ли вектор\\
\hline
T *Begin() const&Функция-член, возвращающая указатель на первый элемент вектора в памяти\\
\hline
T *End() const&Функция-член, возвращающая указатель на элемент, следующий за последним, в векторе в памяти\\
\hline
TVector(const TVector \&other)&Конструктор копирования\\
\hline
TVector \&operator=(const TVector \&other)&Оператор копирования\\
\hline
$\sim{}$TVector()&Деструктор класса\\
\hline
void Clear()&Функция-член очистки объекта\\
\hline
void PushBack(T \&element)&Функция-член добавления элемента в конец вектора(по ссылке)\\
\hline
void PushBack(T \&\&element)&Функция-член добавления элемента в конец вектора(по r-value)\\
\hline
const T \&At(size\_t index) const&Константная функция-член доступа к элементам данной коллекции\\
\hline
T \&At(size\_t index) &Функция-член доступа к элементам данной коллекции\\
\hline
const T \&operator[](size\_t index)const &Константный оператор доступа к элементам данной колекции(аналогично \texttt{At})\\
\hline
T \&operator[](size\_t index)&Оператор доступа к элементам данной колекции(аналогично \texttt{At})\\
\hline
\rowcolor{lightgray}
\multicolumn{2}{|c|} {pair.h}\\
\hline
class TPair&Класс, реализующий тип данных \enquote{пара}\\
\hline
TPair()&Конструктор по-умолчанию\\
\hline
TPair(F first, S second)&Конструктор, явно указывающий элементы пары\\
\hline
TPair(const TPair \&other)&Конструктор копирования\\
\hline
$\sim{}$TPair() = default&Деструктор класса(по-умолчанию)\\
\hline
bool operator<(const TPair \&other) const&Оператор сравнения \enquote{меньше}\\
\hline
TPair \&operator=(const TPair \&other)&Оператор копирования\\
\hline
F first&Первый элемент пары\\
\hline
S second&Второй элемент пары\\
\hline
\rowcolor{lightgray}
\multicolumn{2}{|c|} {sort.h}\\
TKey &Тип ключа\\
\hline
TValue &Тип значения\\
\hline
NPair::TPair<TKey, TValue> MaxElement(const NVector::TVector<NPair::TPair<TKey, TValue>> \&vector)&Возвращает максимальный элемент вектора(по ключу)\\
\hline
NPair::TPair<TKey, TValue> MinElement(const NVector::TVector<NPair::TPair<TKey, TValue>> \&vector)&Возвращает минимальный элемент вектора(по ключу)\\
\hline
void BucketSort(NVector::TVector$<$ NPair::TPair$<$ TKey, TValue$>$$>$ \&vector)&Карманная сортировка\\
\hline
void InsertionSort(NVector::TVector$<$ NPair::TPair$<$TKey, TValue $>$$>$ \&vector)&Сортировка вставками\\
\hline
\end{longtable}

\pagebreak

\section{Консоль}
\begin{alltt}
sakost$ g++ -pedantic -Wall -std=c++17 -Werror -Wno-sign-compare lab1.cpp vector.h pair.h sort.h -o lab1
sakost$ cat test1 
1 4
2 7
1 8
sakost$ ./lab1 < test1 
1 4
1 8
2 4
\end{alltt}
\pagebreak

