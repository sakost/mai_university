\section{Выводы}

Выполнив четвертую лабораторную работу по курсу \enquote{Дискретный анализ}, 
я узнал, как работают алгоритмы поиска образца в строке, а именно \texttt{КМП}.
\\\\
Данный алгоритм лично для меня является одним из самых интуитивно-понятных алгоритмов поиска образца в строке за линейное время.
Также, используемая в данном контексте $\pi$-функция имеет и множество применений в других алгоритмах, в том числе, и других алгоритмах поиска образцов в строке.
Более того, я выделил для себя некоторые хорошие стороны алгоритма:
\begin{enumerate}
    \item Нет ограничений на алфавит
    \item \label{online}Алгоритм является, что называется, \enquote{онлайновым}(об этом ниже)
    \item Работает за линейное время
    \item Реализуется в несколько десятков строк кода
\end{enumerate}
Хочу выделить пункт \ref{online} -- данный пункт означает, что алгоритму не требуются сразу все данные на ввод и он может подсчитывать вхождения непосредственно при вводе текста.
Такая оптимизация достигается путем подсчета префикс-функции только для \enquote{образца} и разделяющего элемента,
а затем в цикле обрабатывается каждый элемент последовательности из текста(и данная обработка происходит один и только один раз),
что мне показалось очень практичным.
\pagebreak
