\section{Описание}
Требуется написать реализацию поиска образца в строке, используя алгоритм Кнута-Морриса-Пратта с алфавитом в виде чисел от $0$ до $2^{32}-1$.

Для начала исходная строка конкатенируется с образцом в таком формате: $p + \# + s$.
Основная идея в поиске подстрок данного алгоритма заключается в том, что предварительно считается, т.н. \enquote{префиксная функция}.
Идея данной функции заключается в том, что на $i$-й позиции она содержит максимально возможное число $k$, такое, что
$s[0\dots{}k-1] = s[i-k+1\dots{}i]$, где $s$ - исодная строка.
Имея исходную строку $p + \# + s$, мы заключаем, что наша задача сводится к задаче поиска значнеия элемена в префикс-функции,
которое будет равно длине образца.

В наивном алгоритме префикс-функции будет учавствовать, по меньшей мере, 2 цикла и одно сравнение строк, что не очень эффективно ($O(n^3)$).
В ходе некоторых рассуждений, можно заключить, что $\pi[i+1]$ никак не может превысить число $\pi[i] + 1$,
что позволяет нам уже добиться сложности $O(n^2)$.

Затем, можно заметить, что если $s[i+1] = s[\pi[i]]$, то мы можем с уверенностью сказать, что $\pi[i+1] = \pi[i] + 1$.
Однако же если $s[i+1] \neq s[\pi[i]]$, тогда следует попытаться подобрать подстроку меньшей длины, которая будет такой наибольшей, что $j<\pi[i]$,
и будет по-прежнему выполняться префикс-свойство в позиции $i$ ($s[0\dots{}j-1] = s[i-j+1\dots{}i]$).

Формула для нахождения такого $j$: $j=\pi[j-1]$. Причем применять её следует, пока $s[i] \neq s[j]$, либо пока $j > 0$(в ином случае префикс-функция в данной позиции равна $0$).
\pagebreak

\section{Исходный код}
Файл \texttt{main.cpp}:
\begin{lstlisting}[language=C++]
	#include <iostream>
	#include <sstream>
	#include <string>
	#include <vector>
	
	using namespace std;
	
	using TValue = uint64_t;
	const TValue shebang = (1ull << 32ull);
	
	// todo: can replace with getline with three args
	vector<TValue> GetClause() {
		TValue c;
		vector<TValue> cl;
		std::string line;
		if(!std::getline(cin, line)){
			throw exception();
		}
		std::istringstream iss(line);
		while ( iss >> c) {
			cl.push_back(c);
		}
		return cl;
	}
	
	void BuildZFunction(vector<TValue> &pattern, vector<TValue> &pi){
		for (size_t i = 1; i < pattern.size(); ++i) {
			TValue j = pi.at(i-1);
			while(j > 0 && pattern.at(i) != pattern.at(j)){
				j = pi.at(j-1);
			}
			if(pattern.at(i) == pattern.at(j)) j++;
			pi.at(i) = j;
		}
	}
	
	
	int main() {
		ios_base::sync_with_stdio(false);
		cin.tie(nullptr); cout.tie(nullptr);
	
		vector<TValue> pattern = GetClause();
		pattern.push_back(shebang);
		vector<TValue> pi(pattern.size(), 0);
		BuildZFunction(pattern, pi);
	
		vector<size_t> rowsLength;
	
		size_t wordInd(0), rowInd(0);
		TValue curPi = 0;
	
		while(true){
			vector<TValue> row;
			try {
				row = GetClause();
			} catch (exception& err) {
				break;
			}
	//        if(row.empty()) continue;
			for (auto& el: row) {
				while(curPi > 0 && el != pattern.at(curPi)){
					curPi = pi.at(curPi - 1);
				}
				if(el == pattern.at(curPi)) ++curPi;
				if(curPi == pattern.size()-1){
					size_t ro = rowInd + 1, wo = wordInd + 1;
					int64_t ind = -1;
					size_t pattern_size = pattern.size()-1;
					while(pattern_size > wo){
						pattern_size -= wo;
						ro--;
						wo = *next(rowsLength.end(), ind--);
					}
					wo -= pattern_size - 1;
					cout << ro << ", " << wo << '\n';
				}
				wordInd++;
			}
			wordInd = 0;
			rowInd++;
			rowsLength.push_back(row.size());
		}
	}
	
\end{lstlisting}

\pagebreak

\section{Консоль}
\begin{alltt}
	sakost@sakost-pc ~/university/2 course/diskran/lab2 ‹master*›$ cmake-build-debug/lab4
	1 1 1 2 1
	1 1 1 2 1 1 1 2 1
	1 1 2
	0 1 2 1 10
	1
	
	1
	
	

	1 2
	
	
	1
	0
	1, 1
	1, 5
	4, 1
	
	sakost@sakost-pc ~/university/2 course/diskran/lab2 ‹master*›$ cmake-build-debug/lab4
	1
	1 2 1 2 1 1
	
	1
	
	
	0
	1
	1, 1
	1, 3
	1, 5
	1, 6
	3, 1
	7, 1
\end{alltt}
\pagebreak

