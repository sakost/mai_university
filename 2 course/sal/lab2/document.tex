\documentclass{article}

% set font encoding for PDFLaTeX, XeLaTeX, or LuaTeX
\usepackage{ifxetex,ifluatex}
\if\ifxetex T\else\ifluatex T\else F\fi\fi T%
\usepackage{fontspec}
\else
\usepackage[T1]{fontenc}
\usepackage[utf8]{inputenc}
\usepackage{lmodern}
\fi

\usepackage{hyperref}
\usepackage[russian]{babel}

\usepackage{sagetex}

% \title{Title of Document}
% \author{Name of Author}

% Enable SageTeX to run SageMath code right inside this LaTeX file.
% http://doc.sagemath.org/html/en/tutorial/sagetex.html
% \usepackage{sagetex}

% Enable PythonTeX to run Python – https://ctan.org/pkg/pythontex
% \usepackage{pythontex}

\begin{document}
	% \maketitle
	
	\begin{center}
		\Large{\textbf{Лабораторная работа 2. Задание 3 -- привидение поверхности второго порядка к каноническому виду.}}
	\end{center}
	
	
	\section{Задание}
	\begin{enumerate}
		\item Привести поверхность, заданную уравнением, к каноническому виду.
		\item Построить исходную поверхность и поверхность в каноническом виде.
		\item Собственные числа и вектора рассчитать вручную, сравнить с результатом встроенных функций.
	\end{enumerate}
	
	\begin{center}
		Вариант 7.
	\end{center}
	$$ 8*x^2 - 2*x*y - 4*y^2 + 2*x*z - 2*y*z + 3*z^2 + 7*x + 8*y + 9*z - 10 $$
	
	\section{Построение исходной поверхности}
	
	\begin{sagesilent}
		var("x y z")
	\end{sagesilent}
	
	\begin{sageblock}
		f(x, y, z) =  8*x**2 - 2*x*y - 4*y**2 + 2*x*z - 2*y*z + 3*z**2 + 7*x + 8*y + 9*z - 10
	\end{sageblock}
	
	Выведем считанную функцию на экран:
	
	\begin{center}
		$\sage{f(x=x, y=y, z=z)}$
	\end{center}
	
	Построим исходную поверхность.
	
	\begin{center}
		\sageplot{implicit_plot3d(f(x=x, y=y, z=z), (x, -30, 10), (y, -30, 10), (z, -10, 30), figsize=3)}
	\end{center}
	
	
\end{document}
