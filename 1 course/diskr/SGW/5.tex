\documentclass[a4paper, 12pt]{article}
\usepackage[left=17mm, top=17mm, right=17mm, bottom=0mm, headsep=1em]{geometry} % лист а4, 12 кегль, тип документа - статья
\usepackage[utf8]{inputenc}  % кодировка вводимого текста
\usepackage[english, russian]{babel}  % подключение словарей с переносами англ и рус яз
\usepackage{amssymb, latexsym, amsmath}  % пакеты для работы с мат символами
\usepackage{indentfirst}  %  каждый абзац с красной строки
\setlength{\parindent}{4ex}
\linespread{0.4} % межстрочный интервал
 
\usepackage{graphicx}
\usepackage{bm}
\usepackage{enumitem}

\usepackage{fancyhdr}

\newcommand{\RNum}[1]{\uppercase\expandafter{\romannumeral #1\relax}}

\pagestyle{fancy}
\fancyhf{}
\rhead{Саженов Константин Станиславович}
\lhead{Группа М8О-108Б-19}
\chead{Вариант 22}
% \rfoot{Page \thepage}
\setlength{\headheight}{28pt}
% \set

\begin{document}
\section*{Задание \RNum{5}}
\paragraph{Текст задания}
Определить, являются ли полем или кольцом векторы размерности 3 с элементами из R и с
операциями поэлементного сложения и
умножения. Проверить, существуют ли делители нуля.
\paragraph{Решение}
$(M, +, \times)$, $M \in \mathbb{R}^{1\times3}$

$ A = \begin{pmatrix}
    a \\ b \\ c
\end{pmatrix}$

\begin{enumerate}
    \item $'+'$ \begin{enumerate}
        \item $ A + B = B + A $ - коммутативность(по свойствам векторов)
        \item $ A + (B + C) = (A + B) + C $ - ассоциативность(по свойствам векторов)
        \item $ e_{+} = \begin{pmatrix}
            0 \\ 0 \\ 0
        \end{pmatrix} \in M$
        \item $ A_{+}^{-1} = -A = \begin{pmatrix}
            -a \\ -b \\ -c
        \end{pmatrix}$
        \item $ A + B \in M $ - замкнутость
    \end{enumerate}
    \item $'\cdot' $ (без $e_{+}$) \begin{enumerate}
        \item $ A \cdot B \in M $ -  замкнутость
        \item $ (A \cdot B) \cdot C = A \cdot (B \cdot C) $ - ассоциативность(св-ва векторов)
        \item $ e_{\times} = \begin{pmatrix}
            1 \\ 1 \\ 1
        \end{pmatrix} \in M $
        \item $ A \cdot B = B \cdot A $ - коммутативность
        \item $ A_{\times}^{-1} $ вектора не $ \exists $. $ \begin{pmatrix}
            1 \\ 0 \\ 0
        \end{pmatrix} $ - не обратима
    \end{enumerate}
    \item Дистрибутивность \\ $ a \cdot (b + c) = \begin{pmatrix}
        x_a \\ y_a \\ z_a
    \end{pmatrix}  \cdot \begin{pmatrix}
        x_b + x_c \\ y_b + y_c \\ z_b + z_c
    \end{pmatrix} = \begin{pmatrix}
        x_a x_b + x_a x_c \\ y_a y_b + y_a y_c \\ z_a z_b + z_a z_c
    \end{pmatrix} = ab + ac$ - в силу коммутативности умножения достаточно одного тождества
    \item Делители нуля \\ $ \begin{pmatrix}
        1 \\ 0 \\ 0
    \end{pmatrix} \cdot \begin{pmatrix}
        0 \\ 1 \\ 0
    \end{pmatrix} = e_{+} $
\end{enumerate}
\paragraph{Ответ:} данная структура образует коммутативное кольцо с делителями нуля 

\end{document}
