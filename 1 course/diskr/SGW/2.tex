\documentclass[a4paper, 12pt]{article}
\usepackage[left=17mm, top=17mm, right=17mm, bottom=0mm, headsep=1em]{geometry} % лист а4, 12 кегль, тип документа - статья
\usepackage[utf8]{inputenc}  % кодировка вводимого текста
\usepackage[english, russian]{babel}  % подключение словарей с переносами англ и рус яз
\usepackage{amssymb, latexsym, amsmath}  % пакеты для работы с мат символами
\usepackage{indentfirst}  %  каждый абзац с красной строки
\setlength{\parindent}{4ex}
\linespread{0.4} % межстрочный интервал
 
\usepackage{graphicx}
\usepackage{bm}
\usepackage{enumitem}

\usepackage{fancyhdr}

\newcommand{\RNum}[1]{\uppercase\expandafter{\romannumeral #1\relax}}

\makeatletter
\AddEnumerateCounter{\asbuk}{\russian@alph}{щ}
\makeatother

\pagestyle{fancy}
\fancyhf{}
\rhead{Саженов Константин Станиславович}
\lhead{Группа М8О-108Б-19}
\chead{Вариант 22}
% \rfoot{Page \thepage}
\setlength{\headheight}{28pt}
% \set

\begin{document}
\section*{Задание \RNum{2}}
\paragraph{Текст задания}
Для заданной подстановки из S8 определить:
\begin{enumerate}[label=\asbuk*)]
    \item разложение в произведение независимых циклов;
    \item порядок подстановки;
    \item разложение в произведение транспозиций;
    \item четность подстановки.
\end{enumerate}
$ [(152)(37)(23578)(162)]^{-115} $
\paragraph{Ответ}
\begin{enumerate}[label=\asbuk*)]
    \item $ [(152)(37)(23578)(162)]^{-115} = [(1678)(253)]^{-115} = (1678)^{-3}(253)^{-1} = (1678)(235)$
    \item $ P = lcm(4,3) = 12 $ - порядок подстановки
    \item Произведение транспозиций: $ (18)(17)(16)(25)(23) $
    \item $ \varepsilon_{\pi} = 5$  - нечетная
\end{enumerate}
\end{document}
