\documentclass[a4paper, 12pt]{article}
\usepackage[left=17mm, top=17mm, right=17mm, bottom=0mm, headsep=1em]{geometry} % лист а4, 12 кегль, тип документа - статья
\usepackage[T2A]{fontenc}
\usepackage[utf8]{inputenc}  % кодировка вводимого текста
\usepackage[english, russian]{babel}  % подключение словарей с переносами англ и рус яз
\usepackage{amssymb, latexsym, amsmath}  % пакеты для работы с мат символами
\usepackage{indentfirst}  %  каждый абзац с красной строки
\setlength{\parindent}{4ex}
\linespread{0.4} % межстрочный интервал
 
\usepackage{graphicx}
\usepackage{bm}
\usepackage{enumitem}
\usepackage{array}
\makeatletter
\AddEnumerateCounter{\asbuk}{\russian@alph}{щ}
\makeatother

\usepackage{fancyhdr}
\usepackage{setspace}

\newcommand{\RNum}[1]{\uppercase\expandafter{\romannumeral #1\relax}}

\pagestyle{fancy}
\fancyhf{}
\rhead{Саженов Константин Станиславович}
\lhead{Группа М8О-108Б-19}
\chead{Вариант 22}
% \rfoot{Page \thepage}
\setlength{\headheight}{28pt}

\begin{document}
\section*{Задание \RNum{1}}
\subsection*{Текст задания} Получить заданную функцию с помощью оператора примитивной рекурсии, используя
оператор суперпозиции, а также функции $ S(x) = x + 1, O(x) = 0, I_m^n(x_1, ..., x_n) = x_m $(где $ 1 \le m \le n$), 
$ \sigma(x_1, x_2) = x_1 + x_2 $.

\subsection*{Решение}
\noindent $ f(x, y) = (x + y)^2 $
\singlespacing
\noindent
$ f(0, y) = y^2 $, \\
$ f(x+1, y) = (x + y + 1)^2 = (x + y + 1)(x + y + 1) = x^2 + 2xy + y^2 + 2x + 2y + 1 = (x+y)^2 + 2x + 2y + 1 =
z + 2x + 2y + 1 = \sigma(\sigma(\sigma(S(x), x), \sigma(y,y)), z) = \\%
= \sigma(\sigma(\sigma(S(I_1(x, y, z)), I_1(x, y, z)), \sigma(I_2(x, y, z), I_2(x, y, z))), I_3(x, y, z))
$
\singlespacing \noindent
$ f(x, 0) = x^2 $, \\ %
$ f(x, y+1) = (x+y+1)^2 = z + 2x + 2y + 1 = \sigma(\sigma(\sigma(S(x), x), \sigma(y,y)), z) = \\ %
= \sigma(\sigma(\sigma(S(I_1(x, y, z)), I_1(x, y, z)), \sigma(I_2(x, y, z), I_2(x, y, z))), I_3(x, y, z))$
\end{document}
