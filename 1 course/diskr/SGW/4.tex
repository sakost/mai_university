\documentclass[a4paper, 12pt]{article}
\usepackage[left=17mm, top=17mm, right=17mm, bottom=0mm, headsep=1em]{geometry} % лист а4, 12 кегль, тип документа - статья
\usepackage[T2A]{fontenc}
\usepackage[utf8]{inputenc}  % кодировка вводимого текста
\usepackage[english, russian]{babel}  % подключение словарей с переносами англ и рус яз
\usepackage{amssymb, latexsym, amsmath}  % пакеты для работы с мат символами
\usepackage{indentfirst}  %  каждый абзац с красной строки
\setlength{\parindent}{4ex}
\linespread{0.4} % межстрочный интервал
 
\usepackage{graphicx}
\usepackage{bm}
\usepackage{enumitem}
\usepackage{array}
\makeatletter
\AddEnumerateCounter{\asbuk}{\russian@alph}{щ}
\makeatother

\usepackage{fancyhdr}

\newcommand{\RNum}[1]{\uppercase\expandafter{\romannumeral #1\relax}}

\pagestyle{fancy}
\fancyhf{}
\rhead{Саженов Константин Станиславович}
\lhead{Группа М8О-108Б-19}
\chead{Вариант 22}
% \rfoot{Page \thepage}
\setlength{\headheight}{28pt}

\begin{document}
\section*{Задание \RNum{4}}
\subsection*{Текст задания} \begin{center} Код Хэмминга(4-7 код). Для слова а) определить соответствующее ему кодовое слово.
Пусть при приеме каждого из слов б), в) возможно была допущена ошибка (не более чем в одной позиции). Определить наличие и положение ошибки.
Какие слова были переданы? Какие слова были закодированны?
\begin{enumerate}[label=\asbuk*),ref=\asbuk*]
    \item 1011
    \item 0001011
    \item 1100011
\end{enumerate}
\end{center}
\subsection*{Решение} 
\paragraph*{a) \RNum{1}-й способ}
\begin{align*}
      M_{3\times7} = \begin{pmatrix}
          0 & 0 & 0 & 1 & 1 & 1 & 1\\
          0 & 1 & 1 & 0 & 0 & 1 & 1\\
          1 & 0 & 1 & 0 & 1 & 0 & 1
      \end{pmatrix}
\end{align*}
$ a = \begin{matrix}
    1 & 0 & 1 & 1
\end{matrix} \\ 
b =\begin{matrix}
    b_1 & b_2 & 1 & b_4 & 0 &1 &1 
\end{matrix} \\
b_3 = a_1 = 1 \\
b_5 = a_2 = 0 \\ 
b_6 = a_3 = 1 \\ 
b_7 = a_4 = 1$
\\
$ bM^T = O \Rightarrow \begin{pmatrix}
    b_1 & b_2 & 1 & b_4 & 0 & 1 & 1
\end{pmatrix} \begin{pmatrix}
    0 & 0 & 1 \\
    0 & 1 & 0 \\
    0 & 1 & 1 \\
    1 & 0 & 0 \\
    1 & 0 & 1 \\
    1 & 1 & 0 \\
    1 & 1 & 1
\end{pmatrix} = O \Leftrightarrow \begin{cases}
    b_4 + b_5 + b_6 + b_7 = 0 \\
    b_2 + b_3 + b_6 + b_7 = 0 \\
    b_1 + b_3 + b_5 + b_7 = 0
\end{cases} \Leftrightarrow \\ \Leftrightarrow \begin{cases}
    b_4 + 0 + 1 + 1 = 0 \\
    b_2 + 1 + 1 + 1 = 0 \\
    b_1 + 1 + 0 + 1 = 0
\end{cases} \Leftrightarrow \begin{cases}
    b_4 = 0 \\ 
    b_2 = 1 \\ 
    b_1 = 0
\end{cases}$ \\
$ b = \begin{matrix}
    0 & 1 & 1 & 0 & 0 & 1 & 1
\end{matrix}$ \\
\paragraph*{\RNum{2}-й способ}
$ b = aC \Leftrightarrow \begin{pmatrix}
    1 & 0 & 1 & 1
\end{pmatrix} \begin{pmatrix}
    1 & 1 & 1 & 0 & 0 & 0 & 0 \\
    1 & 0 & 0 & 1 & 1 & 0 & 0 \\
    0 & 1 & 0 & 1 & 0 & 1 & 0 \\
    1 & 1 & 0 & 1 & 0 & 0 & 1 
\end{pmatrix} = \begin{pmatrix}
    0 & 1 & 1 & 0 & 0 & 1 & 1
\end{pmatrix}$
\paragraph*{б)}
$ c = \begin{matrix}
    0 & 0 & 0 & 1 & 0 & 1 & 1
\end{matrix} \\ 
cM^T = \begin{pmatrix}
    0 & 0 & 0 & 1 & 0 & 1 & 1
\end{pmatrix}\begin{pmatrix}
    0 & 0 & 1 \\
    0 & 1 & 0 \\
    0 & 1 & 1 \\
    1 & 0 & 0 \\
    1 & 0 & 1 \\
    1 & 1 & 0 \\
    1 & 1 & 1
\end{pmatrix} = \begin{pmatrix}
    1 & 0 & 1
\end{pmatrix} $ - ошибка в пятой позиции $ b = \begin{matrix}
    0 & 0 & 0 & 1 & 1 & 1 & 1
\end{matrix} $ - кодовое слово \\ $ a = \begin{matrix}
    0 & 1 & 1 & 1
\end{matrix} $ - исходное сообщение
\paragraph{в)} $ c = \begin{pmatrix}
    1 & 1 & 0 & 0 & 0 & 1 & 1
\end{pmatrix} \begin{pmatrix}
    0 & 0 & 1 \\
    0 & 1 & 0 \\
    0 & 1 & 1 \\
    1 & 0 & 0 \\
    1 & 0 & 1 \\
    1 & 1 & 0 \\
    1 & 1 & 1
\end{pmatrix} = \begin{pmatrix}
    0 & 1 & 0
\end{pmatrix}$ - ошибка во второй позиции \\ $ b = \begin{matrix}
    1 & 0 & 0 & 0 & 0 & 1 & 1
\end{matrix} $ - кодовое слово \\ 
$ a = \begin{matrix}
    0 & 0 & 1 &
\end{matrix} $ - исходное сообщение
\end{document}
