\documentclass[a4paper, 12pt]{article}
\usepackage[left=17mm, top=17mm, right=17mm, bottom=0mm, headsep=1em]{geometry} % лист а4, 12 кегль, тип документа - статья
\usepackage[utf8]{inputenc}  % кодировка вводимого текста
\usepackage[english, russian]{babel}  % подключение словарей с переносами англ и рус яз
\usepackage{amssymb, latexsym, amsmath, mathtext, bm, gensymb, amssymb}  % пакеты для работы с мат символами
\usepackage{indentfirst}  %  каждый абзац с красной строки
\setlength{\parindent}{4ex}
\linespread{0.4} % межстрочный интервал
 
\usepackage{graphicx}
\usepackage{bm}
\usepackage{enumitem}
% \usepackage[T2A]{fontenc}

\usepackage{fancyhdr}

\newcommand{\RNum}[1]{\uppercase\expandafter{\romannumeral #1\relax}}

\makeatletter
\AddEnumerateCounter{\asbuk}{\russian@alph}{щ}
\makeatother

\pagestyle{fancy}
\fancyhf{}
\rhead{Саженов Константин Станиславович}
\lhead{Группа М8О-108Б-19}
\chead{Вариант 22}
% \rfoot{Page \thepage}
\setlength{\headheight}{28pt}
% \set

% variables
\newcommand{\E}{\begin{pmatrix}
    1 & 0 & 0 & 0 \\
    0 & 1 & 0 & 0 \\
    0 & 0 & 1 & 0 \\
    0 & 0 & 0 & 1
\end{pmatrix}} % E matrix

\newcommand{\A}{\begin{pmatrix}
    0 & 1 & 1 & 1 \\
    0 & 0 & 1 & 0 \\
    0 & 1 & 0 & 0 \\
    1 & 1 & 1 & 0
\end{pmatrix}} % A matrix

\newcommand{\Apowtwo}{\begin{pmatrix}
    1 & 1 & 1 & 0 \\
    0 & 1 & 0 & 0 \\
    0 & 0 & 1 & 0 \\
    0 & 1 & 1 & 1
\end{pmatrix}} % A ^ 2

\newcommand{\T}{\begin{pmatrix}
    1 & 1 & 1 & 1\\
    0 & 1 & 1 & 0\\
    0 & 1 & 1 & 0\\
    1 & 1 & 1 & 1
\end{pmatrix}} % T

\newcommand{\Smatrix}{\begin{pmatrix}
    1 & 0 & 0 & 1\\
    0 & 1 & 1 & 0\\
    0 & 1 & 1 & 0\\
    1 & 0 & 0 & 1
\end{pmatrix}} % S


\newcommand{\Sone}{\begin{pmatrix}
    0 & 0 & 0 & 0\\
    0 & 1 & 1 & 0\\
    0 & 1 & 1 & 0\\
    0 & 0 & 0 & 0
\end{pmatrix}} % S_1

\newcommand{\Omatrix}{\begin{pmatrix}
    0 & 0 & 0 & 0\\
    0 & 0 & 0 & 0\\
    0 & 0 & 0 & 0\\
    0 & 0 & 0 & 0
\end{pmatrix}} % S_2 == O

\newcommand{\K}{
    \begin{pmatrix}
        0 & 0 & 0 & 1\\
        0 & 0 & 1 & 0\\
        0 & 1 & 0 & 0\\
        1 & 0 & 0 & 0
    \end{pmatrix}
}


\begin{document}
\section*{Задание \RNum{1}}
\paragraph{Текст задания} 
Определить для орграфа, заданной матрицей смежности:
$$ A = \begin{pmatrix}
    0 & 1 & 1 & 1 \\
    0 & 0 & 1 & 0 \\
    0 & 1 & 0 & 0 \\
    1 & 1 & 1 & 0
\end{pmatrix} $$
\begin{enumerate}[label=\asbuk*)]
    \item матрицу односторонней связности
    \item матрицу сильной связности
    \item компоненты сильной связности
    \item матрицу контуров
\end{enumerate}
\paragraph{Решение}
\begin{enumerate}[label=\asbuk*)]
    \item Найдем матрицу односторонней связности по формуле $ T = E \vee A \vee A^2 \vee A^3$:
    $$ E = \E $$
    $$
    A = \A
    $$
    $$
    A^2 = \A \ast \A = \Apowtwo
    $$
    $$
    A^3 = \Apowtwo \ast \A = \A
    $$

    $$
    T = \E \vee \A \vee \Apowtwo \vee \A = \T
    $$
    $ T = \T$ - матрица односторонней связности
    \item Найдем матрицу сильной связности по формуле $ S = T \& T^T$:
    $$ \bar{S} = \T \& 
    \begin{pmatrix}
        1 & 0 & 0 & 1\\
        1 & 1 & 1 & 1\\
        1 & 1 & 1 & 1\\
        1 & 0 & 0 & 1
    \end{pmatrix} = \Smatrix
    $$
    $ \bar{S} = \Smatrix$ - матрица сильной связности
    \newpage
    \item Компоненты сильной связности:
    $$ \bar{S} = \Smatrix $$
    $$ \{v_1, v_4\} \text{ -- первая компонента сильной связности} $$
    $$ \bar{S}_1  = \Sone $$
    $$ \{v_2, v_3\} \text{ -- вторая компонента сильной связности}$$
    $$ \bar{S}_2 = \Omatrix = O $$
    $$ \bar{S}_2 = O \Rightarrow \bar{S}_2 - \text{нулевая матрица, значит компонент больше нет} $$
    \item Матрица контуров $ K = \bar{S} \& A $
    $$ K = \Smatrix \& \A = \K \Rightarrow \text{дуги:} \left<v_1,v_4\right>, \left<v_4,v_1\right>, \left<v_2,v_3\right>, \left<v_3,v_2\right>$$
\end{enumerate}
\end{document}
