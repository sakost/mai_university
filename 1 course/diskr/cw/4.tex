\documentclass[a4paper, 12pt]{article}
\usepackage[left=17mm, top=17mm, right=17mm, bottom=0mm, headsep=1em]{geometry} % лист а4, 12 кегль, тип документа - статья
\usepackage[utf8]{inputenc}  % кодировка вводимого текста
\usepackage[english, russian]{babel}  % подключение словарей с переносами англ и рус яз
\usepackage{amssymb, latexsym, amsmath, mathtext, bm, gensymb, amssymb}  % пакеты для работы с мат символами
\usepackage{indentfirst}  %  каждый абзац с красной строки
\setlength{\parindent}{4ex}
\linespread{0.4} % межстрочный интервал
 
\usepackage{graphicx}
\graphicspath{ {./images/} }
\usepackage{float}
\usepackage{wrapfig}

\usepackage{bm}
\usepackage{enumitem}
\usepackage[T2A]{fontenc}

\usepackage{fancyhdr}

\newcommand{\RNum}[1]{\uppercase\expandafter{\romannumeral #1\relax}}

\makeatletter
\AddEnumerateCounter{\asbuk}{\russian@alph}{щ}
\makeatother

\pagestyle{fancy}
\fancyhf{}
\rhead{Саженов Константин Станиславович}
\lhead{Группа М8О-108Б-19}
\chead{Вариант 22}
% \rfoot{Page \thepage}
\setlength{\headheight}{28pt}
% \set


\newcommand{\Amatrix}{
    \begin{pmatrix}
        \infty & 2 & 7 & 8 & \infty & \infty & \infty \\
        12 & \infty & 4 & \infty & 6 & \infty & \infty \\
        \infty & 4 & \infty & 1 & 3 & 5 & 7 \\
        \infty & \infty & 1 & \infty & \infty & 3 & \infty \\
        \infty & \infty & 3 & \infty & \infty & \infty & 5 \\
        \infty & \infty & 5 & \infty & \infty & \infty & 2 \\
        2 & \infty & \infty & 3 & 4 & 6 & 7 
    \end{pmatrix}
}


\begin{document}
\section*{Задание \RNum{4}} 
\paragraph{Текст задания} Используя алгоритм Форда, найти минимальные пути из первой вершины во все
достижимые вершины в нагруженном графе, заданном матрицей длин дуг.
$$ A = \Amatrix $$
\paragraph{Решение}
$$\begin{tabular}{ c| c| c |c| c| c| c| c|c| c |c |c |c|c| c| }
    & $v_1$ & $v_2$ & $v_3$ & $v_4$ & $v_5$ & $v_6$ & $v_7$ & $\lambda_i^{(0)}$ & $\lambda_i^{(1)}$ & $\lambda_i^{(2)}$ & $\lambda_i^{(3)}$ & $\lambda_i^{(4)}$ & $\lambda_i^{(5)}$ & $\lambda_i^{(6)}$\\ \hline
    $v_1$ & $\infty$ & 2 & 7 & 8 & $\infty$ & $\infty$ & $\infty$ & 0 & 0 & 0 & 0 & 0 & 0 & 0 \\ \hline
    $v_2$ & 12 & $\infty$ & 4 & $\infty$ & 6 & $\infty$ & $\infty$ & $\infty$ & 2 & 2 & 2 & 2 & 2 & 2\\ \hline
    $v_3$ & $\infty$ & 4 & $\infty$ & 1 & 3 & 5 & 7 & $\infty$ & 7 & 6 & 6 & 6 & 6 & 6 \\ \hline
    $v_4$ & $\infty$ & $\infty$ & 1 & $\infty$ & $\infty$ & 3 & $\infty$ & $\infty$ & 8 & 8 & 7 & 7 & 7 & 7\\ \hline
    $v_5$ & $\infty$ & $\infty$ & 3 & $\infty$ & $\infty$ & $\infty$ & 5 & $\infty$ & $\infty$ & 8 & 8 & 8 & 8 & 8\\ \hline
    $v_6$ & $\infty$ & $\infty$ & 5 & $\infty$ & $\infty$ & $\infty$ & 2 & $\infty$ & $\infty$ & 11 & 11 & 10 & 10 & 10\\ \hline
    $v_7$ & 2 & $\infty$ & $\infty$ & 3 & 4 & 6 & 7  & $\infty$ & $\infty$ & 14 & 13 & 13 & 12 & 12 \\ \hline
    \end{tabular}$$ 
    \begin{enumerate}
        \setcounter{enumi}{1}
        \item Длины минимальных путей из вершины $v_1$ во все остальные вершины определены в последнем столбце таблицы.
        \item Найдем вершины, входящие в минимальные пути из $v_1$ во все остальные вершины графа:
        \begin{enumerate}[label*=\arabic*.]
            \item Минимальный путь из $v_1$ в $v_2$: $v_1 \rightarrow v_2$, его длина -- 2
            $$ \lambda_1^{(0)} + c_{12} = 0 + 2 = \lambda_2^{(1)} $$
            \item Минимальный путь из $v_1$ в $v_3$: $v_1 \rightarrow v_2 \rightarrow v_3$, его длина -- 6
            \begin{align*}
                \lambda_1^{(0)} + c_{13} &= 0 + 7 = 7 = \lambda_3^{(1)} \\
                \lambda_2^{(1)} + c_{23} &= 2 + 4 = 6 = \lambda_3^{(2)}
            \end{align*}
            \item Минимальный путь из $v_1$ в $v_4$: $v_1 \rightarrow v_2 \rightarrow v_3 \rightarrow v_4 $, его длина -- 7
            \begin{align*}
                \lambda_3^{(2)} + c_{34} &= 6 + 1 = 7 = \lambda_4^{(3)} \\
                \lambda_2^{(1)} + c_{23} &= 2 + 4 = 6 = \lambda_3^{(2)} \\
                \lambda_1^{(0)} + c_{12} &= 0 + 2 = 2 = \lambda_2^{(1)}
            \end{align*}
            \item Минимальный путь из $v_1$ в $v_5$: $v_1 \rightarrow v_2 \rightarrow v_5$, его длина -- 8
            \begin{align*}
                \lambda_2^{(1)} + c_{25} &= 2 + 6 = 6 = \lambda_5^{(2)} \\
                \lambda_1^{(0)} + c_{12} &= 0 + 2 = 2 = \lambda_2^{(1)}
            \end{align*}
            \item Минимальный путь из $v_1$ в $v_6$: $v_1 \rightarrow v_2 \rightarrow v_3 \rightarrow v_4 \rightarrow v_6$, его длина -- 10
            \begin{align*}
                \lambda_4^{(3)} + c_{46} &= 7 + 3 = 10 = \lambda_6^{(4)} \\
                \lambda_3^{(2)} + c_{34} &= 6 + 1 = 7 = \lambda_4^{(3)} \\
                \lambda_2^{(1)} + c_{23} &= 2 + 4 = 6 = \lambda_3^{(2)} \\
                \lambda_1^{(0)} + c_{12} &= 0 + 2 = 2 = \lambda_2^{(1)}
            \end{align*}
            \newpage
            \item Минимальный путь из $v_1$ в $v_7$: $v_1 \rightarrow v_2 \rightarrow v_3 \rightarrow v_4 \rightarrow v_6 \rightarrow v_7$, его длина -- 12
            \begin{align*}
                \lambda_6^{(3)} + c_{67} &= 10 + 2 = 12 = \lambda_7^{(5)} \\
                \lambda_4^{(3)} + c_{46} &= 7 + 3 = 10 = \lambda_6^{(4)} \\
                \lambda_3^{(2)} + c_{34} &= 6 + 1 = 7 = \lambda_4^{(3)} \\
                \lambda_2^{(1)} + c_{23} &= 2 + 4 = 6 = \lambda_3^{(2)} \\
                \lambda_1^{(0)} + c_{12} &= 0 + 2 = 2 = \lambda_2^{(1)}
            \end{align*}
        \end{enumerate}
    \end{enumerate}
\end{document}