\documentclass[oneside]{book}

\usepackage[utf8]{inputenc}
\usepackage[left=5mm, top=13mm, right=25mm, bottom=29mm, footskip=10mm]{geometry}
\usepackage{fancyhdr}
\usepackage{amsmath}
\usepackage{amssymb}
\usepackage{gensymb}
\usepackage{bm}
\usepackage{enumitem}
\usepackage{mathtext}
\usepackage[T1, T2A]{fontenc}
\usepackage[english,russian]{babel}

\newcommand{\RNum}[1]{\uppercase\expandafter{\romannumeral #1\relax}}
\pagestyle{fancy}
\fancyhead{}
\fancyhead[L]{\thepage}
\renewcommand{\headrulewidth}{0pt}
\fancyhead[C]{МЕРА. ИЗМЕРИМЫЕ ФУНКЦИИ. ИНТЕГРАЛ}
\fancyhead[R]{ГЛ \RNum{5}}
\fancyfoot{}
\setcounter{page}{280}

\makeatletter
\AddEnumerateCounter{\asbuk}{\russian@alph}{щ}
\makeatother


\fancypagestyle{sec}
{
\fancyhead[R]{\thepage}
\fancyhead[C]{ЛЕБЕГОВО ПРОДОЛЖЕНИЕ МЕРЫ}
\fancyhead[L]{\S 3}
\renewcommand{\headrulewidth}{0pt}
}

\begin{document}
\large
    \textbf{Т е о р е м а 10.} \emph{Если} $A \subset \bigcup_{k=1}^n A_k$, \emph{то} $ \overline{\mu}(A)\leq \sum_{k=1}^n \overline{\mu}(A_k) $.%
    \par\textbf{Т е о р е м а 11.} \emph{Если} $A_k \subset A(k=1, 2, \dots, n)$ \emph{и} $ A_i \cap A_j = \emptyset $\emph{, то}
    \begin{align*}
        \underline{\mu}(A) \geqslant \sum_{k=1}^n \underline{\mu}(A)
    \end{align*}
    \parОпределим теперь функцию $\mu$ на области
    \begin{align*}
        \mathfrak{G}_\mu = \mathfrak{R}^\ast
    \end{align*}
    как общее значение внешней и внутренней меры:
    \begin{align*}
        \mu(A) = \underline{\mu}(A) = \overline{\mu}(A).
    \end{align*}
    \parИз теорем 10 и 11 и из того очевидного обстоятельства, что для $A \in \mathfrak{R}$
    \begin{align*}
        \overline{\mu}(A) = \underline{\mu}(A) = m(A)
    \end{align*}
    вытекает следующее утверждение:
    \par\textbf{Т е о р е м а 12.} \emph{Функция} $\mu(A)$ \emph{является мерой и продолжением меры} $m$.
    \parИзложенное построение применимо  к любой мере $m$, опредеоенной на кольце.
    В частности, его можно применить к множествам на плоскости. При этам за исходное кольцо принимается совокупность элементарных множеств
    (т. е. конечных сумм приямоугольников). Кольцо элементарных множеств зависит, очевидно, от выбора системы координат на плоскости
    (берутся приямоугольники со сторонами, параллельными осями координат). При переходе к плоской мере Жордана эта зависимость от 
    выбора системы координат исчезает: отправляясь от любой системы координат $\{\overline{x}_1, \overline{x}_2\}$, связанной с первоначальной системой
    $\{x_1, x_2\}$ ортогональным преобразованием
    \begin{align*}
        \overline{x}_1 &= \cos\alpha \cdot x_1 + \sin\alpha \cdot x_2 + a_1,\\
        \overline{x}_2 &= -\sin\alpha\cdot x_1 + \cos\alpha\cdot x_2 + a_2,
    \end{align*}
    мы получим одну и ту же меру Жордана. Этот факт вытекает из следующей общей теоремы.
    \par\textbf{Т е о р е м а 13.} \emph{Для того чтобы жордановы продолжения } $\mu_1 = i(m)1$
    \emph{и} $\mu_2 = j(m_2)$ \emph{мер} $m_1$ \emph{и} $m_2$, \emph{определенных на кольцах} $\mathfrak{R}_1$ \emph{и} $\mathfrak{R}_2$
    \emph{совпадали, необходимо и достаточно выполнения условий:}
    \begin{align*}
        \mathfrak{R}_1 \subset \mathfrak{G}_{\mu_2}, m_1(A) &= \mu_2(A) ~на~ \mathfrak{R}_1,\\
        \mathfrak{R}_2 \subset \mathfrak{G}_{\mu_1}, m_2(A) &= \mu_1(A) ~на~ \mathfrak{R}_2.
    \end{align*}
    \par Если исходная мера $m$ определена не на кольце, а на полукольце $\mathfrak{G}_m$, то ее жордановым продолжением естественно назвать меру
    $$j(m) = j(r(m)),$$
    получающуюся в результате продолжения $m$ на кольцо $\mathfrak{R}(\mathfrak{G}_m)$ и дальнейшего продолжения по Жордану.
    \par\textbf{5. Однозначность продолжения меры.} Если множество $A$ измеримо по Жордану относительно меры $\mu$, т.е. принадлежит
    $\mathfrak{R}^\ast = \mathfrak{R}^\ast(\mathfrak{G}_m)$, то для любой меры $\overbrace{\mu}$, продолжающей $m$ и определенной на 
    $\mathfrak{R}^\ast$, значение $\overline{\mu}(A)$ совпадает со значением $J(A)$ жорданова продолжения $J=j(m)$. 
    Можно показать, что продолжение меры $m$ за пределы системы $\mathfrak{R}^\ast$ множеств, измеримых по Жордану, не будет однозначно.
    Более точно это значит следующее. Назовем множество $A$\space\spaceм н о ж е с т в о м\space\spaceо д н о з н а ч н о с т и\space\space для меры $m$,
    если:
    \begin{enumerate}
        \item существует мера, являющаяся продолжением меры $m$, определенная для множества $A$;
        \newpage
        \thispagestyle{sec}
        \item для любых двух такого рода мер $\mu_1$ и $\mu_2$ $$ \mu_1(A) = \mu_2(A) $$
    \end{enumerate}

    \par Имеет место теорема: \emph{система множеств однозначности для меры} $m$ \emph{совпадает с системой множеств, измеримых по Жордану
    относительно меры } $m,$ \emph{т. е. с кольцом} $\mathfrak{R}^\ast.$
    \par Однако если рассматривать только $\sigma$-аддитивные меры и их продолжения($\sigma$-аддитивные),
    то система множеств однозначности будет, вообще говоря, обширнее.
    \par Так как именно случай $\sigma$-аддитивных мер наиболее важен, то введем следующее определение.
    \par О п р е д е л е н и е 6. Множество $A$ называется \emph{множеством $\sigma$-однозначности} для $\sigma$-аддитивной меры $m$, если:
    \begin{enumerate}
        \item существует $\sigma$-аддитивное продолжение $\lambda$ меры $m$, определенное для $A$(т. е. такое, что $A \in \mathfrak{G}_\lambda$);
        \item для всяких таких $\sigma$-аддитивных продолжений $\lambda_1$ и $\lambda_2$ справедливо равенство $$ \lambda_1(A) = \lambda_2(A).$$
    \end{enumerate}
    Если $A$ есть множество $\sigma$-однозначности для $\sigma$-аддитивной меры $\mu$, то в силу нашего определения существует единственно возможное значение $\lambda(A)$
    для любого $\sigma$-аддитивного продолжения меры $\mu$, определенного на $A$.
    \par Легко видеть, что каждое множество $A$, измеримое по Жордану, измеримо и по Лебегу(но не наоборот! приведите пример),
    причем его жорданова и лебегова меры одинаковы. Отсюда непосредственно вытекает, что жорданово продолжение $\sigma$-аддитивной меры $\sigma$-аддитивно.
    \par Каждое множество $A$, измеримое по Лебегу, является множеством $\sigma$-однозначности для исходной меры $m$.
    Действительно, при любом $\varepsilon > 0$ для $A$ существует такое $B \in \mathfrak{R}$, что $\mu^\ast (A \triangle B) < \varepsilon$. 
    Каково бы ни было определенное для $A$ продолжение $\lambda$ меры $m$, $$ \lambda(B) = m'(B), $$
    так как продолжение $m'$ меры $m$ на $\mathfrak{R} = \mathfrak{R}(\mathfrak{G}_m)$ однозначно. Далее,
    $$ \lambda(A \triangle B) \leqslant \mu^\ast(A\triangle B) < \varepsilon $$ и, следовательно, 
    $$ |\lambda(A) - m'(B)| < \varepsilon.$$
    \par Таким образом, для любых двух $\sigma$-аддитивных продолжений $\lambda_1$ и $\lambda_2$ меры $m$ имеем
    $$ |\lambda_1(A) - \lambda_2(A)|<2\varepsilon, $$ откуда в силу произвольности $\varepsilon > 0$
    $$ \lambda_1(A) = \lambda_2(A).$$
    Можно показать, что система множеств, измеримых по Лебегу, исчерпывает всю систему множеств $\sigma$-однозначности для исходной меры $m$.
    \par Пусть $m$ -- некоторая $\sigma$-аддитивная мера с областью определения $\mathfrak{G}$ и $\mathfrak{M} = L(\mathfrak{G})$ -- область определения ее лебегова продолжения.
    Легко убедиться в том, что каково бы ни было полуколько $\mathfrak{G}_1$, удовлетворяющее условию 
    $$ \mathfrak{G} \in \mathfrak{G}_1 \in \mathfrak{M},$$
    всегда $$L(\mathfrak{G}_1) = L(\mathfrak{G}).$$
\end{document}
